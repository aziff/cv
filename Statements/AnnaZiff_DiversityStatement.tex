\documentclass[usenames,dvipsnames,12pt]{article}


\usepackage{amsmath}
 {
      \newtheorem{assumption}{Assumption}
      \newtheorem{lemma}{Lemma}
      \newtheorem{theorem}{Theorem}
  }
  
  \newenvironment{proof}{\paragraph{Proof:}}{\hfill$\square$}

\usepackage{amssymb}
\usepackage{array}

\usepackage[english]{babel}
\usepackage{bm}
\usepackage{booktabs}
\usepackage{caption}
\usepackage{cite}
\usepackage{comment}


\usepackage{float}
\usepackage[margin=1in]{geometry}
\usepackage{graphicx}
\usepackage[dvipsnames]{xcolor}
\usepackage[colorlinks=true,linkcolor=MidnightBlue,urlcolor=BlueViolet,citecolor=BlueViolet]{hyperref}
\usepackage[utf8]{inputenc}
\usepackage{longtable}
\usepackage{natbib}
\usepackage{pdflscape}

\usepackage{sectsty}
\sectionfont{\fontsize{11}{11}\selectfont}
\subsectionfont{\fontsize{11}{11}\selectfont}

\usepackage{setspace}

\usepackage{subcaption}

\usepackage{tabularx}
\newcolumntype{b}{X}
\newcolumntype{s}{>{\hsize=.5\hsize}X}

\usepackage[most]{tcolorbox}
\newtcolorbox{mybox}{breakable,enhanced,colback=black!3!white,colframe=black!15!white}

\usepackage{threeparttable}
\usepackage{tikz}

\usetikzlibrary{shapes,arrows,arrows.meta,trees,positioning}
%\newcommand*{\h}{\hspace{5pt}}% for indentation
%\newcommand*{\hh}{\h\h}% double indentation

\usepackage{titling}
	\setlength{\droptitle}{-4em} % Reduce the vertical space above the title

\usepackage{url}

\begin{document}

\setlength{\parskip}{\baselineskip}%
\setlength{\parindent}{0pt}%
\setlength{\droptitle}{-5em}

\title{\normalsize \textbf{Statement on Diversity, Equity, and Inclusion} \\ \href{https://aziff.github.io}{Anna Ziff} \\ Duke University, Department of Economics}
\date{\vspace{-2.25cm} \normalsize \today \\ \href{https://www.dropbox.com/scl/fi/vt1wcqmt9b7r1oouowp91/AnnaZiff_DiversityStatement.pdf?rlkey=w8rojl4nlqylzhvie7lxkfsf0&dl=0}{Link to Current Version}}
\maketitle

Universities are special institutions in which pluralism can thrive amidst intellectual pursuit and the expansion of opportunity. I aim to contribute to efforts to expand diversity and real inclusion with dedication, humility, and information on institution-specific resources. Once I join a new institution, I plan to familiarize myself with the policies and services of the international and disability offices, as well as other affinity organizations that provide support to those from different backgrounds. 

As an undergraduate student in Chicago, I took coursework and sought out volunteer and work experience to expand my worldview on the realities of race, class, and other sources of social inequality, as well as to positively contribute to my new home. Through Women and Youth Supporting Each Other (WYSE), I volunteered for four years by mentoring middle school girls in a predominantly Chicano and low-income neighborhood. In addition, I interned at two homeless shelters in predominantly Black and low-income neighborhoods. As part of my internship, I traveled throughout the city to learn from politicians, social service providers, academics, and advocates on issues such as affordable housing or access to healthcare. Learning about the historical and political context of these issues and witnessing the widespread disregard (both unintentional and nefarious) for low-income, non-white neighborhoods and their residents underscored to me the necessity for effective policy to produce societal progress and justice. These experiences inform my research agenda of studying the economics of place and place-based policies. I aim to approach this research with maximal seriousness and integrity out of respect for the people potentially affected by the conclusions of the work.


These experiences, as well as experiences in my family, underscore to me the potential for inter-personal connection in increasing inclusion and opportunity, dictating how I approach mentorship in a university setting. First, my experiences as a Latina Jewish woman (both my parents are Jewish and my mother is from Argentina), although not representative, provide some insight into problems that minorities can face in academia. During my time as a graduate student, I mentored at least one first-year Ph.D.\ student per year and two undergraduate students, all from different backgrounds than my own. Given the lower number of women present in economics, it was usually other women who were interested in being my mentee. I find this type of mentor-mentee connection valuable for myself as well to feel connected to other female economists who may face particular challenges within the profession. 

Second, growing up with my autistic brother exposed to me the extra barriers to education for those with disabilities. In addition to the ableism my brother confronts, I witnessed how my parents had to advocate for his inclusion in every educational opportunity. During high school, I volunteered with my brother's disabled community and encouraged my peers to do so as well to positively contribute to that inclusion. These experiences equip me to assist students and colleagues with disabilities, including mental health challenges. As a T.A. during the height of COVID, I met with several undergraduate students who were facing such issues. After speaking with them one-on-one, I shared resources to ensure they were connected to the proper professionals. I am prepared to lend an empathetic ear, seek out the proper resources, and follow up with any assistance that is appropriate and possible. 

Finally, my mother's experiences as an immigrant taught me the real challenges that immigrants can face, especially if English is not their first language.  During my teaching and mentoring experiences, I take care to include international students. For example, I allowed students in my classes and T.A. sections to type their answers over Zoom rather than speak them aloud. I was surprised to see many American students participating through the chat feature. This accommodation is an example of what Haben Girma highlights in her memoir, in which solutions that benefit one group, such as disabled people or immigrants, can extend benefits more broadly. In my next role, I look forward to contributing to the expansion of solutions that contribute to a diverse and inclusive community. 



\end{document}
