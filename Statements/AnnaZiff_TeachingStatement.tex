\documentclass[usenames,dvipsnames,12pt]{article}


\usepackage{amsmath}
 {
      \newtheorem{assumption}{Assumption}
      \newtheorem{lemma}{Lemma}
      \newtheorem{theorem}{Theorem}
  }
  
  \newenvironment{proof}{\paragraph{Proof:}}{\hfill$\square$}

\usepackage{amssymb}
\usepackage{array}

\usepackage[english]{babel}
\usepackage{bm}
\usepackage{booktabs}
\usepackage{caption}
\usepackage{cite}
\usepackage{comment}


\usepackage{float}
\usepackage[margin=1in]{geometry}
\usepackage{graphicx}
\usepackage[dvipsnames]{xcolor}
\usepackage[colorlinks=true,linkcolor=MidnightBlue,urlcolor=BlueViolet,citecolor=BlueViolet]{hyperref}
\usepackage[utf8]{inputenc}
\usepackage{longtable}
\usepackage{natbib}
\usepackage{pdflscape}

\usepackage{sectsty}
\sectionfont{\fontsize{11}{11}\selectfont}
\subsectionfont{\fontsize{11}{11}\selectfont}

\usepackage{setspace}

\usepackage{subcaption}

\usepackage{tabularx}
\newcolumntype{b}{X}
\newcolumntype{s}{>{\hsize=.5\hsize}X}

\usepackage[most]{tcolorbox}
\newtcolorbox{mybox}{breakable,enhanced,colback=black!3!white,colframe=black!15!white}

\usepackage{threeparttable}
\usepackage{tikz}

\usetikzlibrary{shapes,arrows,arrows.meta,trees,positioning}
%\newcommand*{\h}{\hspace{5pt}}% for indentation
%\newcommand*{\hh}{\h\h}% double indentation

\usepackage{titling}
	\setlength{\droptitle}{-4em} % Reduce the vertical space above the title

\usepackage{url}



\setlength{\parskip}{\baselineskip}%
\setlength{\parindent}{0pt}%
\setlength{\droptitle}{-5em}

\usepackage{etoolbox}
\AtBeginEnvironment{quote}{\singlespace\vspace{-\topsep}\small}
\AtEndEnvironment{quote}{\vspace{-\topsep}\endsinglespace}

\begin{document}


\title{\normalsize \textbf{Teaching Statement} \\ \href{https://aziff.github.io}{Anna Ziff} \\ Duke University, Department of Economics}
\date{\vspace{-2.25cm} \normalsize \today \\ \href{https://www.dropbox.com/scl/fi/wudpph2wvd899x0qzkhly/AnnaZiff_TeachingStatement.pdf?rlkey=9dr7o4ngwtfz8iq8jr84s4ea8&dl=0}{Link to Current Version} \\ \href{https://aziff.github.io/teaching/}{Link to Evaluations}}
\maketitle


My teaching experiences and \href{https://gradschool.duke.edu/professional-development/programs/certificate-college-teaching/}{certification in college teaching} have solidified three instructional goals I hope to develop further after graduating. These instructional goals include (1) focusing on student independence and creativity, (2) cultivating an environment of feedback and reflection, and (3) encouraging faculty-student and student-student mentorship relationships. These goals align with the ultimate goal of student learning and development within the economics discipline and apply to both undergraduate and graduate audiences. 

Based on my research experience, I am prepared to teach classes in microeconomics or applied econometrics, especially covering topics in public economics, public finance, urban economics, and labor economics. I plan to use all teaching roles or class design assignments as an opportunity to deepen my own knowledge. Regardless of the particular topic or level, I aim to plan objectives, instruction, activities, and evaluations that support my three core goals. 

\paragraph{Student Independence and Creativity.} Even undergraduates who are not necessarily interested in pursuing graduate school or research careers can benefit from the independence and creativity that the research process cultivates. I designed a \href{https://github.com/aziff/R-Workflow-for-Economists}{graduate-level class on workflow and coding in R} around a capstone project in which students proposed and implemented research projects. I allowed for a replication project rather than an original research project to accommodate students at different levels of preparation. Not only did these activities guide them in developing skills for research, but also provided a more complex understanding of the class material. Integrating independent research into any class I teach with modifications based on the level and topic encourages students to achieve a higher level of learning. For example, in an undergraduate-level class, I would provide a gradual introduction to the iterative process of research, ending with groups of students presenting a research project or replication. Even as a T.A., I shared papers related to the examples of the class. Including independent research projects provides an avenue for all students to challenge themselves, including those who have more experience with the material. 

Incorporating research into the classroom requires covering research ethics and best practices as well. In my class on workflow and coding in R, I designed materials that discussed different approaches to replicable and ethical research. In future teaching opportunities, I will integrate this information into the syllabus.

\paragraph{Feedback and Reflection.} Problem sets, projects, and exams may provide formal evaluation to students, but I aim to cultivate an environment of feedback and reflection beyond these measures. In-class activities and non-credit quizzes reveal information to the students, but also to me, about their gaps in understanding or areas of particular confidence. When I was a T.A., I designed each session to begin with questions that students worked on independently for five minutes. Then, I would either let the students discuss the question in small groups or simply ask students to share their answers with the whole class. This activity allowed students to think through the problem themselves, testing their own knowledge rather than relying on my or another student's explanation. I observed that these non-credit problems were especially important in the undergraduate core econometrics class in which students enter at vastly different levels. In discussing these non-credit problems, one student shared:

\begin{quote}
[Anna\ldots] created some practice problems for us which were VERY helpful for us to understand concepts. As someone who learns by solving problems and in a more ``applied'' way, this teaching style was so helpful and I really appreciate her effort in doing these practice problems for us in office hours because it later helped me to break down the daunting problems in the problem set.
\end{quote}

Prioritizing interactive in-class activities over prolonged lectures results in more continuous feedback and more generally cultivates an environment of reflection. For these activities to work, however, there needs to be a respectful classroom environment. I set this tone from the first day by being clear on classroom expectations (e.g., I start and end class exactly on time). I was pleased to see students highlight this in my evaluations: ``[Anna] never rejects questions nor appears condescending.'' After graduating, to further contribute to this environment, I will have a permanent, anonymous link so that students can submit feedback apart from structured student-instructor evaluations. 

\paragraph{Mentorship Opportunities.} Encouraging instructor-student mentorship relationships expedites students' connection to resources. However, I also aim to encourage student-student mentorship relationships to maximize the students' classroom experiences. Specifically, for some in-class activities, I group students together. As the semester unfolds, these intra-student connections can provide support in students seeking additional help and bolstering student confidence. In order for student-student mentorship to succeed, there must be an emphasis on inclusion. Recent student evaluations mentioned: ``[Anna] created a welcoming environment in her discussion section so that people still felt comfortable asking and answering questions about new topics.'' 


\paragraph{Summary of Teaching Experience.}

I was a T.A. for the undergraduate core class in econometrics at Duke taught by Federico Bugni and James Roberts. Every week I held one office hours and two T.A.\ sessions. For each of these, I prepared additional practice problems to review. I assisted with grading exams. 
\begin{itemize}
    \item \href{https://www.dropbox.com/s/kzoov8nko2mhg5t/Ziff_Anna_Econ%20104.pdf?dl=0}{Evaluations, Fall 2020}
    \item \href{https://www.dropbox.com/s/kjijwsarvrrdj5s/Ziff_Anna_Econ%20204.pdf?dl=0}{Evaluations, Spring 2021}
\end{itemize}

I designed and was the instructor of record for a Ph.D. summer course on coding in \texttt{R}. I took advantage of the opportunity to design a class that also incorporated workflow elements, including \texttt{git} and using computational resources. In addition to holding class, I held weekly office hours. I designed the materials, homeworks, and rubrics. I completed all grading. I taught the course again at Clemson University in the Summer of 2022.

\begin{itemize}
    \item \href{https://github.com/aziff/R-Workflow-for-Economists}{Notes and Class Materials}
    \item \href{https://www.dropbox.com/s/f8fystdpnxc6don/Ziff_Anna_Econ%20890.pdf?dl=0}{Evaluations, Summer 2021}
\end{itemize}

I completed coursework and peer observation through Duke's Certificate of College Teaching. I took two classes: \href{https://gradschool.duke.edu/professional-development/programs/certificate-college-teaching/coursework-teaching/gs750/}{Fundamentals of College Teaching} and \href{https://gradschool.duke.edu/professional-development/programs/certificate-college-teaching/coursework-teaching/gs755-college/}{College Teaching and Course Design}. In addition, I completed a workshop about online teaching to build my skills at hybrid course design and instruction. These resources allowed me to maximize my teaching experiences during graduate school and equipped me with tools and resources for future teaching opportunities.















\end{document}
