\documentclass[usenames,dvipsnames,12pt]{article}


\usepackage{amsmath}
 {
      \newtheorem{assumption}{Assumption}
      \newtheorem{lemma}{Lemma}
      \newtheorem{theorem}{Theorem}
  }
  
  \newenvironment{proof}{\paragraph{Proof:}}{\hfill$\square$}

\usepackage{amssymb}
\usepackage{array}

\usepackage[english]{babel}
\usepackage{bm}
\usepackage{booktabs}
\usepackage{caption}
\usepackage{cite}
\usepackage{comment}


\usepackage{float}
\usepackage[margin=1in]{geometry}
\usepackage{graphicx}
\usepackage[dvipsnames]{xcolor}
\usepackage[colorlinks=true,linkcolor=MidnightBlue,urlcolor=BlueViolet,citecolor=BlueViolet]{hyperref}
\usepackage[utf8]{inputenc}
\usepackage{longtable}
\usepackage{natbib}
\usepackage{pdflscape}

\usepackage{sectsty}
\sectionfont{\fontsize{11}{11}\selectfont}
\subsectionfont{\fontsize{11}{11}\selectfont}

\usepackage{setspace}

\usepackage{subcaption}

\usepackage{tabularx}
\newcolumntype{b}{X}
\newcolumntype{s}{>{\hsize=.5\hsize}X}

\usepackage[most]{tcolorbox}
\newtcolorbox{mybox}{breakable,enhanced,colback=black!3!white,colframe=black!15!white}

\usepackage{threeparttable}
\usepackage{tikz}

\usetikzlibrary{shapes,arrows,arrows.meta,trees,positioning}
%\newcommand*{\h}{\hspace{5pt}}% for indentation
%\newcommand*{\hh}{\h\h}% double indentation

\usepackage{titling}
	\setlength{\droptitle}{-4em} % Reduce the vertical space above the title

\usepackage{url}


\setlength{\parskip}{\baselineskip}%
\setlength{\parindent}{0pt}%
\setlength{\droptitle}{-5em}

\begin{document}

\title{\normalsize \textbf{Research Statement} \\ \href{https://aziff.github.io}{Anna Ziff} \\ Duke University, Department of Economics}
\date{\vspace{-2.25cm} \normalsize \today \\ \href{https://www.dropbox.com/scl/fi/cpkgsn0q5n055d53jussa/AnnaZiff_ResearchStatement.pdf?rlkey=ylj2w4sumidwasqk08gfi3p7t&dl=0}{Link to Current Version}}
\maketitle

My research comprises topics in public economics related to the importance of place and its potential to determine and mediate economic outcomes. My current research agenda and the research I aim to accomplish after graduation contain \textbf{three streams}. First, I study the effectiveness and social efficiency of place-specific investment, considering the responses of households and firms. Second, I study the role of place in economic outcomes of households. For both streams, I study related econometric and measurement issues and pursue separate methodological projects when appropriate. Third, I published three papers as a pre-doctoral researcher studying the provision of early childhood education. Below, I number descriptions of my current and planned projects.

\section{The Effectiveness and Efficiency of Place-Specific Investment}

Within this stream, there are two sets of projects I am currently working on. The first set includes work contained in and related to my job market paper \citep{ziff_local_2023}. 


\begin{enumerate}
\item Job Market Paper: \textit{Beyond the Local Impacts of Place-Based Policies: Spillovers through Latent Housing Markets}.
\end{enumerate}

It is common to study the direct effectiveness of place-based policies, which target geographic areas often to foster economic development. I propose an approach to study the indirect impacts due to responses of households and firms that propagate within similar markets (e.g., housing markets). To determine latent market structure, I apply a network theory method to data on household movement. I illustrate the approach and discuss the economic framework using a widespread, place-based policy, Tax Increment Financing (TIF). With the data-driven characterization, I estimate the ``market'' spillover effects to non-targeted areas within the same housing market. 

For example, TIF can provide a subsidy for retail development of a large vacant lot. The property values of residential properties immediately surrounding the retail development may increase in response to the development. Because these properties are targeted, I consider this to be a direct treatment effect. In addition, there may be a (smaller) positive effect on non-targeted properties that are close to the development. This is the spatial spillover effect. However, the retail development may affect areas within the same housing markets as the targeted areas. Housing markets may not be contiguous due to housing supply, location of amenities, segregation patterns, or other context-specific factors that determine housing markets. If the subsidy relocates investment away from areas within the same housing markets, the values of those non-targeted properties may decrease. This is an example of the market spillover effect that my approach studies and it explains how it can be negative alongside a locally effective policy.

This example is consistent with my empirical findings, which indicate that TIF is locally effective at increasing property values within the targeted area. However, using my approach, I estimate that the market spillover effects on non-targeted areas within the same housing markets are negative. This implies that the policy relocates investment that otherwise would have occurred elsewhere. I analyze outcomes related to household and firm characteristics and find support for the relocation mechanism. 

I combine the direct and spillover effects to calculate a back-of-the-envelope estimate of an overall effect that is close to zero. With caveats, this result suggests that the policy has a low overall effectiveness from the perspective of a central regional government. I compare the characteristics targeted areas to those of the non-targeted areas in the same housing markets. The targeted areas are relatively disadvantaged suggesting some degree of redistribution within housing markets. These results suggest that TIF is costly from the perspective of the central planner despite achieving moderate redistribution within housing markets that already share many observed and unobserved characteristics. \textbf{This is a working paper.}

There are two related methodological considerations that I plan to pursue as separate projects and publications.

\begin{enumerate}
\item[2.] \textit{Measurement of Property Values}. While sale prices reflect the true market price, they are only observed for properties that sell, introducing transaction bias. In contrast, assessed values are observed for all properties, but they may contain non-classical prediction error. In my job market paper, I rely on assessed values for the primary analysis with some supplementary work discussing the possible pitfalls using a measurement model. I plan to expand on this model using nation-wide data in a stand-alone paper that not only considers the tradeoff between prediction error and transaction bias, but also relates the tradeoff to household-level outcomes and equity concerns such as unequal property tax burdens. \textbf{This is an early-stage project.}

\item[3.] \textit{Unifying Panel Data Models for Unobserved Heterogeneity}. I use an event study framework to identify the parameters of interest in my job market paper. I considered several approaches to account for unobserved, time-varying, unit-level heterogeneity. The presence of such heterogeneity challenges the parallel trends assumption. Different models of fixed effects, including grouped fixed effects and interactive fixed effects \citep{bai_panel_2009,bonhomme_grouped_2015}, and changes to the parallel trends assumption can account for the heterogeneity. In a co-authored project \citep{shea_unifying_2023}, we use a bilinear programming approach to unify these models for the purpose of specifying alternative parallel trends assumptions. Bilinear programming allows for identification in the presence of small $T$. This expands the applications for which researchers can use these fixed effects alongside difference-in-difference or event study frameworks. \textbf{This project is in progress.}
\end{enumerate} 

My approach to consider an overall effect combines estimates of direct and spillover effects. I plan to explore another approach that applies Empirical Welfare Maximization (EWM) to consider if place-based policies are properly targeted.

\begin{enumerate}
\item[4.] \textit{Empirical Welfare Maximization for Place-Based Policies}. EWM \citep{kitagawa_who_2018} is a method well-suited for experiments targeted to certain populations. EWM allows one to consider if the program would have been more effective if the eligibility criteria had been different. There is potential to use this method to consider if place-based policies are targeted effectively. However, the challenges of considering spillovers and non-random policy implementation make the problem difficult. I aim to consider how to apply EWM to place-based policies. \textbf{This is an early-stage project.}
%For example, there could be eligibility critera for a job-training program such that only those who are not college graduates can particpate.
\end{enumerate}

The second set of projects relates to co-authored work that studies the response of households to infrastructure investment, and how that response changes the effectiveness of the investment \citep{vinnakota_levees_2023}. 


\begin{enumerate}
\item[5.] \textit{Levees: Infrastructure and Insurance as Adaptation to Flood Risk}.
\end{enumerate}

We study how the take-up of flood insurance changes after the construction of a levee. Both flood insurance and a levee are publicly funded and it is of interest how these two interventions detract from or reinforce each other, given households’ behavioral responses. In addition to observing construction dates, we incorporate novel data on accreditation dates, after which properties protected by accredited levees experience a reduction in the price of flood insurance. \textbf{This project is a working paper.}

In this initial study, we measure to what extent levee infrastructure crowds out households’ insurance take-up. It is documented that regardless of infrastructure, the flood insurance program is undersubscribed, despite insurance premia that are largely below actuarially fair values. Behavioral biases and asymmetric information are possible causes of the low take-up of flood insurance. Given this, the extent to which insurance crowd-out after infrastructure provision is efficient depends on how households update their beliefs on flood risk, relative to the changes in true flood risk. This begets two natural next steps for a research agenda.

\begin{enumerate}
\item[6.] \textit{Household Information on Levees and Flood Insurance}. Our empirical results highlight the counter-intuitive finding that households reduce flood insurance take-up after a levee’s accreditation lowers the cost of flood insurance. It is possible that households underestimate the risk of flooding, which the levee does not completely eradicate. We plan to pursue funding and partnership with government agencies to implement an experiment. We hope to randomly disseminate information about local levee infrastructure and local insurance requirements. Treatment effects on changes in the uptake of flood insurance would reveal to what degree incorrect information determines households’ risky flood insurance choices. \textbf{This is an early-stage project.}
\item[7.] \textit{Enforcement of Flood Insurance Mandates}. Federal law requires that households in flood-prone areas purchase flood insurance. However, the take-up of flood insurance varies widely across the U.S. One challenge in studying this uneven take-up is measuring the enforcement of the mandate, which generally falls on mortgage lenders and the Federal Reserve. We plan to explore different variation determining flood insurance take-up including lender heterogeneity and variation in mandate regulation policies. \textbf{This is an early-stage project.}
\end{enumerate}

% We study how the take-up of flood insurance changes after the construction of a levee. Both flood insurance and a levee are publicly funded and it is of interest how these two interventions detract from or reinforce each other, given households' behavioral responses. In addition to observing construction dates, we incorporate novel data on accreditation dates, after which properties protected by accredited levees experience a reduction in the price of flood insurance. 

% \begin{enumerate}
% \item[5.] We plan to submit the core of this paper, including a description of the data collection of accreditation dates, the empirical estimates on insurance take-up, and a hedonic model, to measure the willingness to pay for flood protection in the presence of heavily subsidized flood insurance.
% \end{enumerate}

% There are two projects that relate to the policy context of levees and flood insurance that continue to interrogate how households respond to the incentives and disincentives tied to living in flood-prone areas. For both of these projects, as well as for future projects in this research agenda, we plan to work to expand the available data on levee construction and accreditation. The data collection we undertook already allow us to observe the accreditation dates of some levees. We aim to expand this further to better study the complex policy environment and economic responses of flooding. 

% \begin{enumerate}
% \item[6.] \textit{Enforcement of Flood Insurance Mandates}. Federal law requires that households in flood-prone areas purchase flood insurance. However, the take-up of flood insurance varies widely across the U.S. One challenge in studying this uneven take-up is measuring the enforcement of the mandate, which generally falls on mortgage lenders and the Federal Reserve. We plan to explore different variation determining flood insurance take-up including lender heterogeneity and variation in mandate regulation policies. 

% \item[7.] \textit{Household Information on Levees and Flood Insurance}. Our empirical results highlight the counter-intuitive finding that households reduce flood insurance take-up after a levee's accreditation lowers the cost of flood insurance. It is possible that households underestimate the risk of flooding, which the levee does not completely eradicate. We plan to pursue funding and partnership with government agencies to implement an experiment. We hope to randomly disseminate information about levee protectiveness and local insurance requirements. Treatment effects on changes in the uptake of flood insurance would reveal to what degree incorrect information determines households' risky flood insurance choices.
% \end{enumerate}


\section{The Importance of Place for Household Outcomes}


The first stream of my research agenda relates to economic framework and econometric approaches to consider the consequences of place-based interventions; place can also play a role in economic outcomes even without a specific intervention tied to it. The second stream of my research agenda relates to these circumstances.

\begin{enumerate}

\item[8.] \textit{Rental Choice Sets in Low- and High-Opportunity Neighborhoods for Housing Choice Voucher Program Participants}. Participants in the Housing Choice Voucher Program (HCVP) face constraints in location choice both due to constrained resources and additional policy requirements of the program. A challenge to understand these constraints is the measurement of the choice sets. In \citet{park_rental_2023}, we combine restricted-access data from the U.S.\ Department of Housing and Urban Development (HUD) with a novel data source on advertised rental prices to observe the weekly choice sets of HCVP participants, and describe them by geographic location and household demographics. Observing the choice sets expands modeling possibilities in a structural framework of residential sorting. \textbf{This project is in progress.}


\item[9.] \textit{Association between the Volatility of Income and Life Expectancy in the U.S.} In \citet{ziff_association_2023}, we study the relationship between household income volatility and life expectancy using a household-level consumer dataset. We find that income is an important mediator of the correlation, with the bottom half of the income distribution experiencing negative correlation between income volatility and life expectancy, including if the income volatility is positive. The role of place is notable both in the geographic distribution of households in the bottom half of the income distribution and the possibility that place-specific factors, such as factory closings, contribute to this correlation. \textbf{This paper is Revise and Resubmit at the \textit{Journal of Labor Economics}.}


\end{enumerate}

\section{The Provision of Early Childhood Education}

Before graduate school, I worked as a pre-doctoral researcher with Professor James Heckman during which I co-authored three projects related to early childhood education. Each project enhanced my understanding of the process of economics research, revealed my specific interests and skills, and provided experience with the publication process. 

\begin{enumerate}

    \item[10.] \textit{Evaluation of the Reggio Approach to Early Education.} The non-experimental nature of our approach involved collecting qualitative information on the roll-out of a program, exposing me to the practice of consulting sources outside of economics to bolster an empirical framework \citep{biroli_evaluation_2018}. \textbf{This paper is published in \textit{Research in Economics}.}

    \item[11.] \textit{Early Childhood Education and Crime.} We study the gender differences of the effects of early childhood education on criminal activity. Although early childhood education has a larger effect on female criminal activity, the social benefit of reducing male criminal activity is larger due to gender differences in the types of criminal activity \citep{garcia_gender_2018}. \textbf{This paper is published in \textit{European Economic Review}.}

    \item[12.] \textit{Gender Differences in the Benefits of an Influential Early Childhood Program.} Building on the gender differences reported in \citet{garcia_gender_2018}, we use combining functions to adequately estimate gender differences in treatment effect outcomes for subjects from randomized control trials \citep{garcia_early_2019}. \textbf{This paper is published in \textit{Infant Mental Health Journal}.}

\end{enumerate}

\singlespacing
\bibliographystyle{chicago}
\bibliography{../CV/AnnaZiff.bib, ../TIF.bib}


\end{document}
