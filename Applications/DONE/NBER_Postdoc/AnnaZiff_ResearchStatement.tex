\documentclass[usenames,dvipsnames,12pt]{article}


\usepackage{amsmath}
 {
      \newtheorem{assumption}{Assumption}
      \newtheorem{lemma}{Lemma}
      \newtheorem{theorem}{Theorem}
  }
  
  \newenvironment{proof}{\paragraph{Proof:}}{\hfill$\square$}

\usepackage{amssymb}
\usepackage{array}

\usepackage[english]{babel}
\usepackage{bm}
\usepackage{booktabs}
\usepackage{caption}
\usepackage{cite}
\usepackage{comment}


\usepackage{float}
\usepackage[margin=1in]{geometry}
\usepackage{graphicx}
\usepackage[dvipsnames]{xcolor}
\usepackage[colorlinks=true,linkcolor=MidnightBlue,urlcolor=BlueViolet,citecolor=BlueViolet]{hyperref}
\usepackage[utf8]{inputenc}
\usepackage{longtable}
\usepackage{natbib}
\usepackage{pdflscape}

\usepackage{sectsty}
\sectionfont{\fontsize{11}{11}\selectfont}
\subsectionfont{\fontsize{11}{11}\selectfont}

\usepackage{setspace}

\usepackage{subcaption}

\usepackage{tabularx}
\newcolumntype{b}{X}
\newcolumntype{s}{>{\hsize=.5\hsize}X}

\usepackage[most]{tcolorbox}
\newtcolorbox{mybox}{breakable,enhanced,colback=black!3!white,colframe=black!15!white}

\usepackage{threeparttable}
\usepackage{tikz}

\usetikzlibrary{shapes,arrows,arrows.meta,trees,positioning}
%\newcommand*{\h}{\hspace{5pt}}% for indentation
%\newcommand*{\hh}{\h\h}% double indentation

\usepackage{titling}
	\setlength{\droptitle}{-4em} % Reduce the vertical space above the title

\usepackage{url}


\setlength{\parskip}{\baselineskip}%
\setlength{\parindent}{0pt}%
\setlength{\droptitle}{-5em}

\begin{document}

\title{\normalsize \textbf{Research Proposal} \\ \href{https://aziff.github.io}{Anna Ziff} \\ Duke University, Department of Economics}
\date{\vspace{-2.25cm} \normalsize \today }
\maketitle


It is common to study the direct effectiveness of place-based policies, which target geographic areas often to foster economic development. In my job market paper, I propose an approach to study the indirect impacts due to responses of households and firms that propagate within similar markets (e.g., housing markets). To determine latent market structure, I apply a network theory method to data on household movement. I illustrate the approach and discuss the economic framework using a widespread, place-based policy, Tax Increment Financing (TIF). With the data-driven characterization, I estimate the ``market'' spillover effects to non-targeted areas within the same housing market. 

For example, TIF can provide a subsidy for retail development of a large vacant lot. The property values of residential properties immediately surrounding the retail development may increase in response to the development. Because these properties are targeted, I consider this to be a direct treatment effect. In addition, there may be a (smaller) positive effect on non-targeted properties that are close to the development. This is the spatial spillover effect. However, the retail development may affect areas within the same housing markets as the targeted areas. Housing markets may not be contiguous due to housing supply, location of amenities, segregation patterns, or other context-specific factors that determine housing markets. If the subsidy relocates investment away from areas within the same housing markets, the values of those non-targeted properties may decrease. This is an example of the market spillover effect that my approach studies and it explains how it can be negative alongside a locally effective policy.

This example is consistent with my empirical findings, which indicate that TIF is locally effective at increasing property values within the targeted area. However, using my approach, I estimate that the market spillover effects on non-targeted areas within the same housing markets are negative. This implies that the policy relocates investment that otherwise would have occurred elsewhere. I analyze outcomes related to household and firm characteristics and find support for the relocation mechanism. 

I combine the direct and spillover effects to calculate a back-of-the-envelope estimate of an overall effect that is close to zero. With caveats, this result suggests that the policy has a low overall effectiveness from the perspective of a central regional government. I compare the characteristics targeted areas to those of the non-targeted areas in the same housing markets. The targeted areas are relatively disadvantaged suggesting some degree of redistribution within housing markets. These results suggest that TIF is costly from the perspective of the central planner despite achieving moderate redistribution within housing markets that already share many observed and unobserved characteristics. 

I propose to pursue the following extensions of my job market paper and work on them as stand-alone projects.

\begin{enumerate}
\item[1.] \textit{Measurement of Property Values}. While sale prices reflect the true market price, they are only observed for properties that sell, introducing transaction bias. In contrast, assessed values are observed for all properties, but they may contain non-classical prediction error. In my job market paper, I rely on assessed values for the primary analysis with some supplementary work discussing the possible pitfalls using a measurement model. I plan to expand on this model using nation-wide data in a stand-alone paper that not only considers the tradeoff between prediction error and transaction bias, but also relates the tradeoff to household-level outcomes and equity concerns such as unequal property tax burdens. 

\item[2.] \textit{Unifying Panel Data Models for Unobserved Heterogeneity}. I use an event study framework to identify the parameters of interest in my job market paper. I considered several approaches to account for unobserved, time-varying, unit-level heterogeneity. The presence of such heterogeneity challenges the parallel trends assumption. Different models of fixed effects, including grouped fixed effects and interactive fixed effects \citep{bai_panel_2009,bonhomme_grouped_2015}, and changes to the parallel trends assumption can account for the heterogeneity. In a co-authored project \citep{shea_unifying_2023}, we use a bilinear programming approach to unify these models for the purpose of specifying alternative parallel trends assumptions. Bilinear programming allows for identification in the presence of small $T$. This expands the applications for which researchers can use these fixed effects alongside difference-in-difference or event study frameworks. 

\item[3.] \textit{Empirical Welfare Maximization for Place-Based Policies}. EWM \citep{kitagawa_who_2018} is a method well-suited for experiments targeted to certain populations. EWM allows one to consider if the program would have been more effective if the eligibility criteria had been different. There is potential to use this method to consider if place-based policies are targeted effectively. However, the challenges of considering spillovers and non-random policy implementation make the problem difficult. I aim to consider how to apply EWM to place-based policies. 
\end{enumerate}


\singlespacing
\bibliographystyle{chicago}
\bibliography{AnnaZiff.bib,TIF.bib}


\end{document}
